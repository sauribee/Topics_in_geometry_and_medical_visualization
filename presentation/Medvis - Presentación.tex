\documentclass[10pt]{beamer}
\mode<presentation>
{
  \usetheme{Antibes}
  \useoutertheme{default}
  \setbeamertemplate{navigationsymbols}{only frame symbol}
}

\usepackage{color}
\usepackage{verbatim}
\usepackage{subfigure}
\usepackage{amsmath,amsthm,amsfonts,amscd,amssymb,amsbsy,epsf}
\usepackage[spanish]{babel}
\usepackage[utf8]{inputenc}
\usepackage{times}
\usepackage{graphics}
\usepackage{bbm}
\usepackage{enumerate}
\usepackage{hyperref}
\definecolor{darkblue}{rgb}{0.2,0,1}
\definecolor{darkred}{rgb}{1,0,0.5}

%%%%%%%%%%%%%%%%%%%%%%%%%%%%%%%%%%%%%%

\title[Curvas de Bézier y B-splines en Modelado Médico de Cráneos]{Modelado Geométrico con Curvas de Bézier y B-splines \\
\small{Aplicación a Contornos de Cráneos}}
\author{Autor. Santiago Uribe Echavarría \texorpdfstring{\\}{, } Profesor. Marco Paluszny Kluczynsky \texorpdfstring{\\}{, } Tópicos en Geometría y Visualización Médica \thanks{Universidad Nacional de Colombia, Sede Medell{\'i}n}}
\date{\small{Repositorio: \url{https://github.com/sauribee/Topics_in_geometry_and_medical_visualization}}}

%%%%%%%%%%%%%%%%%%%%%%%%%%%%%%%%%%%%%% 
\begin{document}
%%%%%%%%%%%%%%%%%%%%%%%%%%%%%%%%%%%%%%
\begin{frame}
  \titlepage
\end{frame}
%%%%%%%%%%%%%%%%%%%%%%%%%%%%%%%%%%%%%%

%%%%%%%%%%%%%%%%%%%%%%%%%%%%%%%%%%%%%%
\begin{frame}
\frametitle{Tabla de contenidos}
  \vspace{-.5cm}
    \tableofcontents
  \vspace{-.5cm}
\end{frame}
%%%%%%%%%%%%%%%%%%%%%%%%%%%%%%%%%%%%%%

%%%%%%%%%%%%%%%%%%%%%%%%%%%%%%%%%%%%%%
\section{Marco teórico}
%%%%%%%%%%%%%%%%%%%%%%%%%%%%%%%%%%%%%%

\begin{frame}
	\frametitle{Curvas de Bézier}

  \begin{columns}
    \begin{column}{0.5\textwidth}
      Las curvas de Bézier son curvas paramétricas definidas por puntos de control. Se basan en los polinomios de Bernstein:
      \[
        \mathbf{C}(t) = \sum_{i=0}^{n} B_{i,n}(t) \mathbf{P}_i, \quad t \in [0,1]
      \]
      donde $B_{i,n}(t) = \binom{n}{i} t^i (1-t)^{n-i}$ son las bases de Bernstein y $\mathbf{P}_i$ los puntos de control.
      
      \begin{itemize}
        \item[-] Pasan por el primer y último punto de control
        \item[-] Control intuitivo mediante puntos de control
        \item[-] Grado $n-1$ para $n$ puntos 
      \end{itemize}
    \end{column}

    \begin{column}{0.5\textwidth}
      \begin{figure}
        \centering
        \includegraphics[width=1\textwidth]{../reports/figures/bezier_bsplines_reports/bezier_interpolation_curves/01_círculo_interpolation.png}
        \caption{Interpolación de Bézier en círculo}
      \end{figure}
    \end{column}
  \end{columns}
\end{frame}

\begin{frame}
	\frametitle{B-splines}

  \begin{columns}
    \begin{column}{0.5\textwidth}
      Los B-splines son curvas paramétricas por tramos que ofrecen control local. Se definen mediante:
      \[
        \mathbf{S}(t) = \sum_{i=0}^{n} N_{i,p}(t) \mathbf{P}_i
      \]
      donde $N_{i,p}(t)$ son las funciones base B-spline de grado $p$ y $\mathbf{P}_i$ los puntos de control.
      
      \textbf{Ventajas sobre Bézier:}
      \begin{itemize}
        \item[-] Mejor control local
        \item[-] Estabilidad numérica para muchos puntos
        \item[-] Grado fijo independiente del número de puntos
      \end{itemize}
    \end{column}

    \begin{column}{0.5\textwidth}
      \begin{figure}
        \centering
        \includegraphics[width=1\textwidth]{../reports/figures/skull_reports_01/bspline_skull_interpolation/02_protuberancia_bspline.png}
        \caption{Interpolación B-spline en protuberancia}
      \end{figure}
    \end{column}
  \end{columns}
\end{frame}

\begin{frame}
	\frametitle{Parametrización de curvas}

  \begin{columns}
    \begin{column}{0.6\textwidth}
      La parametrización determina cómo se distribuyen los valores de $t$ entre los puntos. Métodos implementados:
      
      \begin{enumerate}
        \item[1.] \textbf{Uniforme}: $t_i = i / (n-1)$
        
        \item[2.] \textbf{Chord-length}: $t_i = t_{i-1} + \|\mathbf{P}_i - \mathbf{P}_{i-1}\|$
        
        \item[3.] \textbf{Centripetal}: $t_i = t_{i-1} + \|\mathbf{P}_i - \mathbf{P}_{i-1}\|^{0.5}$
        
        \item[4.] \textbf{Arc-chord}: combina longitud de arco con cuerda
      \end{enumerate}
      
      \vspace{0.3cm}
      La parameterización chord-length es más estable para curvas irregulares.
    \end{column}

    \begin{column}{0.4\textwidth}
      \begin{figure}
        \centering
        \includegraphics[width=1\textwidth]{../reports/figures/bezier_bsplines_reports/arc_chord_parameterization/99_comparison_grid.png}
        \caption{Comparación de parametrizaciones}
      \end{figure}
    \end{column}
  \end{columns}
\end{frame}

\begin{frame}
	\frametitle{Matriz de Bernstein}

  Dada una parametrización $\{t_0, t_1, \ldots, t_m\}$ y un grado $n$, para la \textbf{matriz de Bernstein} $\mathbf{B} \in \mathbb{R}^{(m+1) \times (n+1)}$ se define la entrada $i,j$ como:
  \[
    B_{i,j} = B_{j,n}(t_i) = \binom{n}{j} t_i^j (1-t_i)^{n-j}
  \]
  y de esta forma podemos decir que
  \[
    C(t_j) = \sum_{i=0}^{n} B_{i,n}(t_j) \mathbf{P}_i
  \]

  y así en forma matricial si $\mathbf{B}$ es la matriz de Bernstein, se tiene que
  \[
    \begin{bmatrix} C(t_1) \\ C(t_2) \\ \vdots \\ C(t_m) \end{bmatrix} = \mathbf{B} \begin{bmatrix} \mathbf{P}_0 \\ \mathbf{P}_1 \\ \vdots \\ \mathbf{P_n} \end{bmatrix} = \begin{bmatrix} B_{0,n}(t_1) & B_{1,n}(t_1) & \cdots & B_{n,n}(t_1) \\ B_{0,n}(t_2) & B_{1,n}(t_2) & \cdots & B_{n,n}(t_2) \\ \vdots & \vdots & \ddots & \vdots \\ B_{0,n}(t_m) & B_{1,n}(t_m) & \cdots & B_{n,n}(t_m) \end{bmatrix} \begin{bmatrix} \mathbf{P}_0 \\ \mathbf{P}_1 \\ \vdots \\ \mathbf{P_n} \end{bmatrix}
  \]

\end{frame}

\begin{frame}
	\frametitle{Aplicación en interpolación/aproximación}

  \textbf{Aplicación en interpolación/aproximación:}

  La matriz $\mathbf{B}$ puede ser utilizada para resolver problemas de interpolación y aproximación como se muestra a continuación:

  \begin{itemize}
    \item[-] \textbf{Interpolación} ($m = n$): Resolver $\mathbf{B} \mathbf{P} = \mathbf{D}$ para un vector de puntos de control $\mathbf{P}$
    \item[-] \textbf{Aproximación LSQ} ($m > n$): Minimizar $\|\mathbf{B} \mathbf{P} - \mathbf{D}\|^2$ mediante $\mathbf{P} = (\mathbf{B}^T \mathbf{B})^{-1} \mathbf{B}^T \mathbf{D}$
  \end{itemize}

  donde $\mathbf{D}$ son los puntos de datos a interpolar/aproximar.

\end{frame}

%%%%%%%%%%%%%%%%%%%%%%%%%%%%%%%%%%%
\section{Aplicación médica}
%%%%%%%%%%%%%%%%%%%%%%%%%%%%%%%%%%%

\begin{frame}
  \frametitle{Planteamiento del problema}

  \textbf{Objetivo:} Modelar contornos de cráneo a partir de datos médicos discretos.

  \vspace{0.3cm}
  \textbf{Datos de entrada:}
  \begin{itemize}
    \item[-] Coordenadas $(x, y)$ de puntos del borde del cráneo
    \item[-] Lado izquierdo y derecho del contorno
    \item[-] Múltiples cortes axiales (10 slices)
  \end{itemize}

  \vspace{0.3cm}
  \textbf{Desafíos:}
  \begin{itemize}
    \item[-] Datos ruidosos e irregularmente espaciados
    \item[-] Necesidad de curvas suaves para visualización
    \item[-] Detección de características anatómicas (protuberancia occipital)
    \item[-] Mantener precisión sin oscilaciones excesivas
  \end{itemize}
\end{frame}

\begin{frame}
  \frametitle{Enfoques de modelado implementados}

  Se implementaron tres enfoques complementarios:

  \vspace{0.3cm}
  \begin{enumerate}
    \item[1.] \textbf{Interpolación de Bézier}: Curva que pasa exactamente por los puntos de control.
    \begin{itemize}
      \item[$\cdot$] Grado = N - 1 (puede ser alto para muchos puntos)
      \item[$\cdot$] Oscilaciones para muchos puntos
    \end{itemize}

    \item[2.] \textbf{Aproximación LSQ de Bézier}: Curva suave que minimiza el error cuadrático.
    \begin{itemize}
      \item[$\cdot$] Grado fijo (2-8), independiente de la cantidad $N$ de puntos
      \item[$\cdot$] Sin oscilaciones, curva suave
    \end{itemize}

    \item[3.] \textbf{Interpolación B-spline}: Combina lo mejor de ambos.
    \begin{itemize}
      \item[$\cdot$] Interpolación exacta con grado fijo (cúbico)
      \item[$\cdot$] Control local, estabilidad numérica
    \end{itemize}
  \end{enumerate}
\end{frame}

\begin{frame}
  \frametitle{Estabilidad numérica}

  \textbf{Problema:} Curvas de Bézier de alto grado sufren inestabilidad numérica.

  \vspace{0.3cm}
  \textbf{Soluciones implementadas:}
  \begin{itemize}
    \item[-] Cálculo en espacio logarítmico para coeficientes binomiales:
    \[
      \log\binom{n}{k} = \log\Gamma(n+1) - \log\Gamma(k+1) - \log\Gamma(n-k+1)
    \]
    donde $\Gamma(\alpha) = \displaystyle\int_0^\infty t^{\alpha-1} e^{-t} dt$, es la función gamma.

    \item[-] SVD para resolver sistemas mal condicionados
    
    \item[-] Advertencias cuando grado $> 15$ (interpolación) o $> 20$ (LSQ)
    
    \item[-] Monitoreo del número de condición de matrices de Bernstein
    
  \end{itemize}

  \vspace{0.3cm}
  \textbf{Resultado:} Estabilidad garantizada.
\end{frame}

%%%%%%%%%%%%%%%%%%%%%%%%%%%%%%%%%%
\section{Metodología y parámetros}
%%%%%%%%%%%%%%%%%%%%%%%%%%%%%%%%%%

\begin{frame}
  \frametitle{Herramientas y tecnologías}

  \begin{columns}
    \begin{column}{0.7\textwidth}
      \textbf{Implementación}:
      \begin{itemize}
        \item[-] \textbf{Lenguaje}: Python 3.12+
        \item[-] \textbf{Bibliotecas numéricas}:
        \begin{itemize}
          \item[$\cdot$] NumPy $\geq$ 2.1 y SciPy $\geq$ 1.14
          \item[$\cdot$] Bezier $\geq$ 2024.6.20
        \end{itemize}
        \item[-] \textbf{Visualización}:
        \begin{itemize}
          \item[$\cdot$] Matplotlib, PyVista, VTK
        \end{itemize}
        \item[-] \textbf{I/O médico}: SimpleITK, pydicom, nibabel
      \end{itemize}
      
      \vspace{0.3cm}
      \textbf{Estructura del proyecto}:
      \begin{itemize}
        \item[-] Paquete instalable: \texttt{medvis}
        \item[-] Scripts de análisis reproducibles
        \item[-] Tests automatizados con pytest
      \end{itemize}
    \end{column}
    
    \begin{column}{0.3\textwidth}
      \begin{figure}
        \centering
        \vspace{0.5cm}
        \textbf{Repositorio}
        
        \vspace{0.3cm}
        \small{\url{github.com/sauribee/Topics_in_geometry_and_medical_visualization}}
      \end{figure}
    \end{column}
  \end{columns}
\end{frame}

\begin{frame}
  \frametitle{Métricas de evaluación}

  \textbf{Error cuadrático medio (MSE):}
  \[
    \text{MSE} = \frac{1}{N} \sum_{i=1}^{N} \|\mathbf{C}(t_i) - \mathbf{P}_i\|^2
  \]

  \vspace{0.3cm}
  \textbf{Error máximo:}
  \[
    \text{Error}_\text{max} = \max_{i=1,\ldots,N} \|\mathbf{C}(t_i) - \mathbf{P}_i\|
  \]

  \vspace{0.3cm}
  \textbf{Para B-splines (interpolación):}
  \begin{itemize}
    \item[-] Error teórico: precisión de máquina ($\sim 10^{-13}$)
    \item[-] Confirmado experimentalmente en todos los casos
  \end{itemize}

  \vspace{0.3cm}
  \textbf{Para aproximación LSQ:}
  \begin{itemize}
    \item[-] Error depende del grado elegido
    \item[-] Trade-off entre suavidad y precisión
  \end{itemize}
\end{frame}

\begin{frame}
  \frametitle{Parámetros de ejecución - skull\_reports\_01}

  \textbf{Dataset:} Corte axial único del cráneo

  \vspace{0.3cm}
  \begin{columns}
    \begin{column}{0.5\textwidth}
      \textbf{Interpolación Bézier:}
      \begin{itemize}
        \item[$\cdot$] Puntos cráneo: 12
        \item[$\cdot$] Puntos protuberancia: 6
        \item[$\cdot$] Umbral Y: 52
        \item[$\cdot$] Parameterización: chord-length
      \end{itemize}
      
      \vspace{0.3cm}
      \textbf{Aproximación LSQ Bézier:}
      \begin{itemize}
        \item[$\cdot$] Puntos muestreados: 14
        \item[$\cdot$] Grado cráneo: 8
        \item[$\cdot$] Grado protuberancia: 5
      \end{itemize}
    \end{column}

    \begin{column}{0.5\textwidth}
      \textbf{Interpolación B-spline:}
      \begin{itemize}
        \item[$\cdot$] Puntos cráneo: 14
        \item[$\cdot$] Puntos protuberancia: 7
        \item[$\cdot$] Grado: 3 (cúbico)
        \item[$\cdot$] Knot vector: uniforme abierto
      \end{itemize}
      
      \vspace{0.3cm}
      \textbf{Preprocesamiento:}
      \begin{itemize}
        \item[$\cdot$] Muestreo uniforme por longitud de arco
        \item[$\cdot$] Detección automática de protuberancia
      \end{itemize}
    \end{column}
  \end{columns}
\end{frame}

\begin{frame}
  \frametitle{Parámetros de ejecución - skull\_reports\_02}

  \textbf{Dataset:} 10 cortes axiales (corte 0 - corte 9)

  \vspace{0.3cm}
  \begin{columns}
    \begin{column}{0.5\textwidth}
      \textbf{Procesamiento por lotes:}
      \begin{itemize}
        \item[$\cdot$] Método: B-spline cúbico
        \item[$\cdot$] Puntos muestreados (full): 20
        \item[$\cdot$] Puntos muestreados (prot): 10
        \item[$\cdot$] Grado: 3
        \item[$\cdot$] Umbral Y: 50
      \end{itemize}
      
      \vspace{0.3cm}
      \textbf{Estadísticas del dataset:}
      \begin{itemize}
        \item[$\cdot$] Promedio: 568 puntos/slice
        \item[$\cdot$] Rango: 418-828 puntos
      \end{itemize}
    \end{column}

    \begin{column}{0.5\textwidth}
      \textbf{Detección de protuberancia:}
      \begin{itemize}
        \item[$\cdot$] 7/10 slices con protuberancia
        \item[$\cdot$] Cortes 0-6: protuberancia detectada
        \item[$\cdot$] Cortes 7-9: solo contorno completo
      \end{itemize}
      
      \vspace{0.3cm}
      \textbf{Outputs generados:}
      \begin{itemize}
        \item[$\cdot$] 10 reportes individuales
        \item[$\cdot$] Grids comparativos (2×5)
        \item[$\cdot$] CSV con métricas
      \end{itemize}
    \end{column}
  \end{columns}
\end{frame}

%%%%%%%%%%%%%%%%%%%%%%%%%%%%%%%%%%
\section{Resultados - skull\_reports\_01}
%%%%%%%%%%%%%%%%%%%%%%%%%%%%%%%%%%

\begin{frame}
  \frametitle{Comparación de métodos - Contorno completo}

  \begin{columns}
    \begin{column}{0.33\textwidth}
      \begin{figure}
        \centering
        \includegraphics[width=1\textwidth]{../reports/figures/skull_reports_01/bezier_skull_interpolation/01_craneo_completo_interpolation.png}
        \caption{Bézier interpolación}
      \end{figure}
    \end{column}

    \begin{column}{0.33\textwidth}
      \begin{figure}
        \centering
        \includegraphics[width=1\textwidth]{../reports/figures/skull_reports_01/bezier_skull_approximation/01_craneo_completo_approximation.png}
        \caption{Bézier LSQ}
      \end{figure}
    \end{column}

    \begin{column}{0.33\textwidth}
      \begin{figure}
        \centering
        \includegraphics[width=1\textwidth]{../reports/figures/skull_reports_01/bspline_skull_interpolation/01_craneo_completo_bspline.png}
        \caption{B-spline}
      \end{figure}
    \end{column}
  \end{columns}
\end{frame}

\begin{frame}
  \frametitle{Comparación de métodos - Protuberancia}

  \begin{columns}
    \begin{column}{0.33\textwidth}
      \begin{figure}
        \centering
        \includegraphics[width=1.25\textwidth]{../reports/figures/skull_reports_01/bezier_skull_interpolation/02_protuberancia_interpolation.png}
        \caption{Bézier interpolación}
      \end{figure}
    \end{column}

    \begin{column}{0.33\textwidth}
      \begin{figure}
        \centering
        \includegraphics[width=1.25\textwidth]{../reports/figures/skull_reports_01/bezier_skull_approximation/02_protuberancia_approximation.png}
        \caption{Bézier LSQ}
      \end{figure}
    \end{column}

    \begin{column}{0.33\textwidth}
      \begin{figure}
        \centering
        \includegraphics[width=1.25\textwidth]{../reports/figures/skull_reports_01/bspline_skull_interpolation/02_protuberancia_bspline.png}
        \caption{B-spline}
      \end{figure}
    \end{column}
  \end{columns}
\end{frame}

\begin{frame}
  \frametitle{Grids comparativos}

  \begin{columns}
    \begin{column}{0.5\textwidth}
      \begin{figure}
        \centering
        \includegraphics[width=1\textwidth]{../reports/figures/skull_reports_01/bezier_skull_interpolation/00_comparison_grid.png}
        \caption{Interpolación Bézier}
      \end{figure}
    \end{column}

    \begin{column}{0.5\textwidth}
      \begin{figure}
        \centering
        \includegraphics[width=1\textwidth]{../reports/figures/skull_reports_01/bspline_skull_interpolation/00_comparison_grid.png}
        \caption{Interpolación B-spline}
      \end{figure}
    \end{column}
  \end{columns}
\end{frame}

%%%%%%%%%%%%%%%%%%%%%%%%%%%%%%%%%%
\section{Resultados - skull\_reports\_02}
%%%%%%%%%%%%%%%%%%%%%%%%%%%%%%%%%%

\begin{frame}
  \frametitle{Análisis multi-slice - Puntos de datos}

  \begin{figure}
    \centering
    \includegraphics[width=1\textwidth]{../reports/figures/skull_reports_02/skull_points/all_slices_grid.png}
    \caption{10 cortes axiales con puntos de contorno originales (418-828 puntos por slice)}
  \end{figure}
\end{frame}

\begin{frame}
  \frametitle{Análisis de protuberancia - Multi-slice}

  \begin{figure}
    \centering
    \includegraphics[width=1\textwidth]{../reports/figures/skull_reports_02/protuberance_analysis/protuberances_comparison_grid.png}
    \caption{Detección y modelado de protuberancia occipital en cortes 0-4}
  \end{figure}
\end{frame}

\begin{frame}
  \frametitle{Conexión de lados - Ejemplos}

  \begin{figure}
    \centering
    \includegraphics[width=1\textwidth]{../reports/figures/skull_reports_02/skull_sides_connection/skull_sides_comparison_grid.png}
    \caption{Conexión suave entre lado izquierdo y derecho en cortes 6-8}
  \end{figure}
\end{frame}

%%%%%%%%%%%%%%%%%%%%%%%%%%%%%%%%%%
\section{Conclusiones}
%%%%%%%%%%%%%%%%%%%%%%%%%%%%%%%%%%

\begin{frame}
  \frametitle{Conclusiones}

  \textbf{Resultados principales:}
  \begin{itemize}
    \item[-] \textbf{B-splines cúbicos}: Método óptimo para contornos médicos
    \begin{itemize}
      \item[$\cdot$] Interpolación exacta (error $\sim 10^{-13}$)
      \item[$\cdot$] Estabilidad numérica garantizada
      \item[$\cdot$] Control local para edición
    \end{itemize}
    
    \item[-] \textbf{Bézier interpolación}: Útil para pocos puntos ($< 10$)
    \begin{itemize}
      \item[$\cdot$] Oscilaciones para muchos puntos
      \item[$\cdot$] Control global más intuitivo
    \end{itemize}
    
    \item[-] \textbf{Bézier LSQ}: Mejor para suavizado
    \begin{itemize}
      \item[$\cdot$] Trade-off entre suavidad y precisión
      \item[$\cdot$] No pasa exactamente por los puntos
    \end{itemize}
  \end{itemize}
\end{frame}

\begin{frame}

  \vspace{1cm}

  \begin{center}
    \textbf{¡Gracias por su atención!}
    
    \vspace{0.75cm}
    \small{Repositorio: \url{https://github.com/sauribee/Topics_in_geometry_and_medical_visualization}}
  \end{center}

\end{frame}

\end{document}